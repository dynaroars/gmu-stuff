\documentclass[11pt]{article}
\usepackage{fullpage}
\usepackage{hyperref}
\usepackage{nopageno}

\usepackage{xcolor}
\newcommand{\alert}[1]{{\color{blue}{#1}}}

\begin{document}

\begin{center}
{\Large \textbf{Facilities, Equipment, and Other Resources}}
\end{center}


\section*{Facilities}

\subsection*{George Mason University (Mason)} Mason is the largest and fastest-growing public university in Virginia and holds an R1 designation from the Carnegie Classification of Institutions of Higher Education. Mason accounts for almost $60\%$ of the growth of higher education in Virginia, enrolling more than  $40$K students in Fall 2023. Mason is also a very diverse institution; the majority of its more than $7,000$ incoming freshmen in Fall 2022 were students of color, with $48\%$ of females, and more than half being first-generation students. In particular, Mason's undergraduate student body diversity ranked 10th among public universities in the US. Mason has a rich infrastructure for diversity and inclusivity across all educational and research programs through various programs, such as Mason LIFE, Social Action \& Integrative Learning (SAIL), the Kellar Institute for Human disAbilities, the Diversity Research Group, the Patriot Experience, and more. Mason has state-of-the-art facilities, offices, laboratories,  conference spaces, extensive library access, and robust computing and storage capabilities.

\alert{The PIs and their labs reside in Mason's College of Science (PI Agnarsson) and College of Engineering and Computing (PIs Nguyen and Liu). Mason's College of Science has more than $5,000$ undergraduate and graduate students and its College of Enginering and Computing houses close to $9,000$. The two colleges are among the largest at Mason and the primary factor behind Mason being the largest producer of tech talent in Northern Virginia.}

% Two of the investigators in this project sit in the Department of Compuer Science, which currently ranks 36nd by CSRankings.org.

 %Important to the diversification plans of the investigators, the colleges have been strategic in its diversification plan and has managed to increase the ethnic diversity for under-represented student groups by 180\% and has increased gender diversity by 42\%. CEC currently ranks $92$nd among the Best Engineering Schools (U.S. News \& World Report Best Graduate Schools, 2021). Five of its graduate programs rank within the top $100$ of their discipline.

\subsection*{\alert{Agnarsson Lab}} Agnarsson's office is located in Mason's Exploratory Hall (room 4412). His workstation is provided by the department of Mathematical Sciences. His graduate students have desks and desktop-mounted workstations in Exploratory Hall and have access to printers and internet connection through the department’s networks.

\subsection*{\alert{Nguyen Lab}} Nguyen's office is located in Mason's Nguyen Engineering building (room \#4404). His lab has two Linux servers, provided by the department of Computer Science, with multicores and fast GPUs to support DNN research. Nguyen currently has five dedicated student desks and workstations in the SWE laboratory in Nguyen Engineering building.

\subsection*{\alert{Liu Lab}} Liu's office is located in Mason's Research Hall building (room 355). His workstation is provided by the department of Computer Science. His graduate students have desks and desktop-mounted workstations in the machine learning laboratory (RSH 357) in Research Hall. Liu currently has three dedicated student desks and workstations in his laboratory.


\section*{Equipment and Computational Resources}

\subsection*{Mason Office of Research Computing (ORC)} The ORC has compute resources that may be provisioned as either virtual or physical host machines for short- and long-term research projects. The ORC has a $7$PB storage cluster from which it can provision highly redundant storage volumes on request to research project systems. ORC staff provide and maintain the operating system and required software on research project systems, such that it complies with Mason system and network security standards. Additionally, ORC staff maintain regular backups of the research system’s data. Firewalls protect university systems against unauthorized access and malicious attacks. Mason is fully connected to the internet with Lambda Rail connectivity, access to grid and cloud computing facilities. have access to a plethora of enterprise-level collaboration and video/audio conferencing tools both on their personal computers, including Zoom, MS Teams, and Blackboard Collaborate Ultra. 	

\subsection*{Mason ORC Hopper Cluster} The PIs and their students have access to the Hopper Cluster \url{https://wiki.orc.gmu.edu/mkdocs/About_HOPPER/}. Hopper is a batch computing resource available to all faculty and their students built using standard OpenHPC 2.x tools. Login nodes specs are: 2x Dell PowerEdge R640, 2x Intel(R) Xeon(R) Gold 6240R CPU @ 2.40GHz, 2xAMD 384 GB DDR4 RAM RHEL8[Rocky Linux release 8.5(green obsidian)]. Compute nodes specs are: 74x Dell PowerEdge R640,
2x Intel(R) Xeon(R) Gold 6240R CPU @ 2.40GHz, 48 cores per node
192 GB DDR4 RAM, 960 GB local SSD storage, RHEL8[Rocky Linux release 8.5(Green obsidian)]. GPU nodes specs are: 1x NVIDIA DGX-A100, 8x NVIDIA A100-SXM4-40GB GPUs, 2x AMD EPYC Rome 7742 CPUs @ 2.60 GHz, 128 cores per node, 1 TB DDR4 RAM, 512GB DDR4 RAM.

\section*{Other Resources Available to the Investigators and their Students}

\subsection*{Mason Graduate Education Division (GED)} GED is a major unit of Academic Affairs within the Office of the Provost, serving all graduate programs and students. Its mission is to elevate graduate education at Mason by increasing the global impact of its graduate students and programs, fostering a collaborative culture of academic excellence, and contributing to the research productivity and the workforce development appropriate to a world-class research institution. GED provides competitive awards to graduate students, as well as co-curricular, professional development programs, often in partnerships with the institutes.

The PIs will make use of these resources for the graduate students participating in this project. These resources also include professional development programs and opportunities to enrich the graduate student experience and better position the students for furthering their careers post graduation.

\subsection*{Mason Institute for Digital InnovAtion (IDIA)} IDIA is a transdisciplinary institute representing Mason’s commitment to shaping the future of our digital society, promote equality, well-being, security, and prosperity for all. The institute offers various research incubation programs for faculty and students, as well as co-curricular training and development programs offered as workshops (often in partnership with GED) for students. These include training in ethical research, best practices in researcher, scholarship, and collaborations, entrepreneurship and translation of knowledge, science writing and communication, as well as networking and mentorship opportunities. The institute holds regular events to connect graduate students with local, state, and regional leaders from industry, government, and non-profits. IDIA also provides opportunities for graduate students to make a broader impact in outreach events (such as bootcamps, hackathons, etc.) targeted at middle school, high-school students, and undergraduate students. Located at the center of the Rosslyn-Ballston corridor, in Fuse at Mason square, IDIA will also provide state-of-the-art lab spaces, classrooms and conference spaces to support the broader impact and broadening participation in computing activities of the team.

The PIs will encourage the graduate students in this project to participate in such events and use these resources to improve their professional development and networking opportunities.

\subsection*{Mason Graduate and Professional Student Association (GAPSA)}  GAPSA is the principal body advocating for the needs of graduate and professional students, and its goal is to enrich the academic and social life for all graduate and professional students.

The PIs will encourage the graduate students in this project to connect through GAPSA with the larger graduate student community so that they can enrich their experiences.

\subsection*{Mason Stearns Center for Teaching and Learning} The Stearns Center for Teaching and Learning is Mason’s hub for promoting teaching excellence and innovation. The Center provides instructional training and is a great resource for graduate students and postdoctoral researchers considering academic careers.

The PIs will encourage their students to make use of the Center's training programs and resources, which include professional development.

\end{document}
